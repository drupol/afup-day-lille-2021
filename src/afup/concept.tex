\begin{frame}{Evaluation}{Les possibilites existantes}
    \begin{itemize}[<+->]
        \item Callback (thunk)
        \item Iterateur / Generateur
    \end{itemize}
\end{frame}

\begin{frame}[fragile]{Evaluation}{traditionnelle}
    \begin{lstlisting}[firstnumber=1]
        <?php

        $array = range('a', 'z');
    \end{lstlisting}
\end{frame}

\begin{frame}[fragile]{Evaluation}{traditionnelle}
    \begin{lstlisting}[firstnumber=1]
        <?php

        $array = range(0, 1000 ** 3);
    \end{lstlisting}
\end{frame}

\begin{frame}[fragile]{Evaluation traditionnelle}{Et bardaf...}
    \begin{spverbatim}
        PHP Fatal error: Allowed memory size of 536870912 bytes exhausted (tried to allocate 34359738376 bytes) in example.php on line 5
    \end{spverbatim}
\end{frame}

\begin{frame}[fragile]{Evaluation}{paresseuse via un thunk}
    \begin{lstlisting}[firstnumber=1]
        <?php

        $array = static fn (): array => range('a', 'z');
    \end{lstlisting}
\end{frame}

\begin{frame}[fragile]{Evaluation}{paresseuse via un thunk}
    \begin{lstlisting}[firstnumber=1]
        <?php

        $array = static fn (): array => range(0, 1000 ** 3);
    \end{lstlisting}
\end{frame}

\begin{frame}{Evaluation}{paresseuse}
        Un \textit{thunk} est une valeur qui est en attente d'evaluation.

        \pause

        Un \textit{thunk} est une function qui enrobe (wrap) une expression,
        pour postposer son evaluation.

        \pause

        Le terme est né comme une version humoristique du passé du verbe
        anglophone \textit{penser} (\textit{to think}).
\end{frame}

\begin{frame}[fragile]{Evaluation}{paresseuse via un generateur}
    \begin{lstlisting}[firstnumber=1]
        <?php

        function xrange($start, $end, $step = 1): Generator {
            for ($i = $start; $i <= $end; $i += $step) {
                yield $i;
            }
        }

        foreach (xrange(0, 1000 ** 3) as $value) {
            // Do stuff
        }
    \end{lstlisting}
\end{frame}

\begin{frame}{Evaluation}{Schématisons une évaluation traditionnelle}
    \newcolumntype{t}{>{\tt}c}
        \begin{center}
            \[
            \left[ \begin{array}{tt}
                0 \Rightarrow a \\
                1 \Rightarrow b \\
                2 \Rightarrow a
            \end{array} \right]
            \pause
            \xrightarrow{\texttt{array\_flip()}}
            \left[ \begin{array}{tt}
                a \Rightarrow 2 \\
                b \Rightarrow 1
            \end{array} \right]
            \pause
            \xrightarrow{\texttt{array\_flip()}}
            \left[ \begin{array}{tt}
                2 \Rightarrow a \\
                1 \Rightarrow b
            \end{array} \right]
            \]%
    \end{center}
\end{frame}

\begin{frame}{Evaluation}{Schématisons une évaluation traditionnelle}
    \newcolumntype{t}{>{\tt}c}
    \begin{center}
            \[
            \left[ \begin{array}{tt}
                0 \Rightarrow \left[ \begin{array}{tt} a \Rightarrow A\end{array} \right] \\
                1 \Rightarrow \left[ \begin{array}{tt} b \Rightarrow B\end{array} \right] \\
                2 \Rightarrow \left[ \begin{array}{tt} c \Rightarrow C\end{array} \right]
            \end{array} \right]
            \pause
            \xrightarrow{\texttt{array\_flip()}}
            \raisebox{-.5\height}{\includegraphics[scale=0.05]{meme/No-RageFace.jpg}}
            \]%
            \texttt{Fatal error: Uncaught TypeError: Illegal offset type}
    \end{center}
\end{frame}

\begin{frame}{Schematisons}{evaluation \textit{lazy}}
    \newcolumntype{t}{>{\tt}c}
    \begin{center}
        \[
            \left[ \begin{array}{tt}
                0 \Rightarrow a \\
                1 \Rightarrow b \\
                2 \Rightarrow a
            \end{array} \right]
            \pause
            \xrightarrow{\texttt{array\_flip()}}
            \left[ \begin{array}{tt}
                \only<2->{a \Rightarrow 0} \\
                \only<4->{b \Rightarrow 1} \\
                \only<6->{a \Rightarrow 2}
            \end{array} \right]
            \pause
            \xrightarrow{\texttt{array\_flip()}}
            \left[ \begin{array}{tt}
                \only<3->{0 \Rightarrow a} \\
                \only<5->{1 \Rightarrow b} \\
                \only<7->{2 \Rightarrow a}
            \end{array} \right]
        \]
    \end{center}
\end{frame}

