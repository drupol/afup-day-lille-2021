\begin{frame}{Une collection}{Une définition simple}
    \begin{itemize}[<+->]
        \item Une collection est une structure unifiée pour représenter et manipuler un ensemble de données
        \item Permettant de les manipuler indépendamment de ce qu'elles contiennent
        \item Elle permettent de réduire l'effort tout en améliorant les performances
        \item Favorise l'interoperabilité et la reutilisation
        \item Inclus des implémentations et algorithmes pour manipuler les données
    \end{itemize}
\end{frame}

\begin{frame}{Collections}{ça existe déjà!}
    \begin{itemize}[<+->]
        \item \texttt{doctrine/collections}
        \item \texttt{voku/Arrayy}
        \item Laravel
        \item CakePHP
    \end{itemize}
\end{frame}

\begin{frame}{Collections}{Pourquoi?}
    \begin{itemize}[<+->]
        \item Unifier la manière de travailler avec des ensemble de données
        \item Optimiser les algorithmes
        \item Faciliter l'emploi de certaines fonctionnalités
    \end{itemize}
\end{frame}

\begin{frame}{Collections}{Problèmes rencontrés?}
    \begin{itemize}[<+->]
        \item La plupart sont des wrapper de fonctions natives PHP
        \item La plupart se fonctionnent qu'avec des \texttt{array}
    \end{itemize}
\end{frame}
