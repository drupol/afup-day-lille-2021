\begin{frame}{Motivations}{Une liste exhaustive}
    \begin{itemize}[<+->]
        \item Par \colorbox{yellow}{\textbf{curiosité}} et envie d'apprendre
        \item Pour \colorbox{yellow}{\textbf{comprendre}} et mettre en pratique
        des concepts appris dans d'autre languages fonctionnels (\textit{Haskell, OCaml})
        \item Pour \colorbox{yellow}{\textbf{propager}} ces concepts et les rendre facilement accessible
        \item Pour \colorbox{yellow}{\textbf{favoriser}} l'interopérabilité et l'\textit{éco-informatique}
    \end{itemize}
\end{frame}

\begin{frame}
    \begin{flushleft}
        \textit{
            Les chercheurs du Centre for Energy-Efficient Telecommunications (CEET) et des laboratoires Bell ont montré que
            les technologies de l'information et de la communication (TIC), qui englobent Internet, les vidéos, les fichiers sonores
            et autres services dans le cloud, produiraient plus de 830 millions de tonnes de CO\textsubscript{2} chaque année.\\
            \bigskip
            Cela représente 2\% des émissions globales du principal gaz à effet de serre.\\
            \bigskip
            Une telle quantité de CO\textsubscript{2} équivaut aux émissions de l'industrie aérienne.
        }
    \end{flushleft}

    \blankfootnote{Source: \href{https://www.futura-sciences.com/planete/actualites/developpement-durable-emissions-co2-liees-internet-polluent-autant-avion-43802/}{Futura Sciences}}
\end{frame}

\begin{frame}
    \begin{flushleft}
        \textit{
            Je fais du vélo le matin, je voudrais ne pas respirer la poussière générée
            par ces datacenters fous fonctionnant si mal avec des algorithmes,
            car on se donne entre les mains un pouvoir qu'il ne maîtrise pas.
        }
        \begin{flushright}
            \tiny{---Julien Pauli}
        \end{flushright}
    \end{flushleft}

    \blankfootnote{Source: \href{http://blog.jpauli.tech/a-new-move-in-my-career/}{Julien Pauli blog}}
\end{frame}

\begin{frame}{loophp/collection}{A propos}
    \begin{itemize}[<+->]
        \item OpenSource, dispo sur Github
        \item Disponible pour PHP >= 7.4
        \item Commencé en Août 2019
        \item Pas de dépendances
        \item $\pm$ 800 commits
        \item Entièrement paresseuse
        \item $\pm$ 85 méthodes disponibles
        \item Testée, typée
    \end{itemize}
\end{frame}

\begin{frame}{loophp/collection}{Idées et concepts}
    \begin{itemize}[<+->]
        \item Basé sur une utilisation exhaustive des générateurs
        \item Concepts directement inspirés du language Haskell
        \item Corrige beaucoup d'incohérences de PHP
        \item Algorithmes optimisés et re-utilisés partout
    \end{itemize}
\end{frame}

\begin{frame}{loophp/collection}{Utilisé par qui?}
    \begin{itemize}[<+->]
        \item Sur des projets personnels
        \item Par des clients en Belgique et Luxembourg
        \item Par des utilisateurs pour agréger des données en France
    \end{itemize}
\end{frame}

\begin{frame}[fragile]{Exemple d'utilisation 0}{Double flip}
    \begin{lstlisting}[firstnumber=1]
    <?php

    Collection::fromIterable(['a', 'b', 'c', 'd', 'a'])
        ->flip()
        ->flip()
        ->all();
    \end{lstlisting}

    \pause

    \begin{lstlisting}[firstnumber=8]
    // ['a', 'b', 'c', 'd', 'a']
    \end{lstlisting}
\end{frame}

\begin{frame}[fragile]{Exemple d'utilisation 1}{On commence en douceur}
    \begin{lstlisting}[firstnumber=1]
    <?php

    Collection::times(1000000)
        ->filter(static fn (int $n): bool => $n % 2 === 0)
        ->filter(static fn (int $n): bool => str_contains((string) $n, '321'))
        ->filter(static fn (int $n): bool => strlen((string) $n) >= 4)
        ->limit(1000);
    \end{lstlisting}

    \pause

    \begin{lstlisting}[firstnumber=9]
    // 3210, 3212, 3214, 3216, 3218, 13210, 13212, ... 493218
    \end{lstlisting}
\end{frame}

\begin{frame}[fragile]{Exemple d'utilisation 2}{Analyse de texte}
    \begin{lstlisting}[firstnumber=1]
    <?php

    $string = file_get_contents('https://loripsum.net/api');
    // $string = <p>Lorem ipsum dolor sit amet, consectetur adipiscing elit. ...</p>

    Collection::fromString(file_get_contents('https://loripsum.net/api'))
        ->filter(static fn (string $i): int => preg_match('/^[a-zA-Z]+$/', $i))
        ->map(static fn (string $i): string => strtolower($i))
        ->frequency()
        ->flip()
        ->sort()
        ->reverse();
    \end{lstlisting}

    \pause

    \begin{lstlisting}[firstnumber=14]
    /**
        [i] => 137
        [e] => 128
        [t] => 96
        [a] => 92
        [u] => 88
        [s] => 77
        ...
    */
    \end{lstlisting}
\end{frame}

\begin{frame}[fragile]{Exemple d'utilisation 3}{La suite de Fibonacci}
    \begin{lstlisting}[firstnumber=1]
    <?php

    Collection::unfold(static fn ($a = 0, $b = 1): array => [$b, $a + $b])
        ->limit(10);
    \end{lstlisting}

    \pause

    \begin{lstlisting}[firstnumber=6]
    /**
        [0, 1],
        [1, 2],
        [2, 3],
        [3, 5],
        [5, 8],
        [8, 13],
        [13, 21],
        [21, 34],
        [34, 55],
        [44, 89],
    */
    \end{lstlisting}
\end{frame}

\begin{frame}[fragile]{Exemple d'utilisation 4}{Approximation du nombre $\pi$}
    \begin{lstlisting}[firstnumber=1]
    <?php

    $precision = 10 ** -4;

    Collection::unfold(static fn (int $n = 0): int => $n + 1)
        ->map(
            static fn (int $i): int => $i * 2,
            static fn (int $i): array => [$i, $i],
            static fn (array $chunk): array => [
                $chunk[0] / ($chunk[0] - 1),
                $chunk[1] / ($chunk[1] + 1)
            ]
        )
        ->unwrap()
        ->reduction(
            static fn (float $carry, float $i): float => $carry * $i, 2
        )
        ->window(1)
        ->drop(1)
        ->until(
            static fn (array $v): bool => abs($v[0] - $v[1]) < ($precision / 2)
        )
        ->last(); // 3.1415676541454
    \end{lstlisting}
\end{frame}

\begin{frame}
    \begin{center}
        {\Large \texttt{\href{https://github.com/loophp/collection}{https://github.com/loophp/collection}}\par}
    \end{center}
\end{frame}
