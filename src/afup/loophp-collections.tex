\begin{frame}{Motivations}{Une liste exhaustive}
    \begin{itemize}[<+->]
        \item Par curiosité et envie d'apprendre
        \item Pour comprendre et mettre en pratique des concepts appris dans
        d'autre languages tels que Haskell
        \item Favoriser la sémantique et propager une bonne utilisation du code
        \item Pour favoriser l'interopérabilité et l'éco-informatique
    \end{itemize}
\end{frame}

\begin{frameC}{Mais aussi\ldots}

\end{frameC}

\begin{frame}
    \begin{flushleft}
        \textit{
            Les chercheurs du Centre for Energy-Efficient Telecommunications (CEET) et des laboratoires Bell ont montré que
            les technologies de l'information et de la communication (TIC), qui englobent Internet, les vidéos, les fichiers sonores
            et autres services dans le cloud, produiraient plus de 830 millions de tonnes de CO\textsubscript{2} chaque année.\\
            Cela représente 2\% des émissions globales du principal gaz à effet de serre.\\
            Une telle quantité de CO\textsubscript{2} équivaut aux émissions de l'industrie aérienne.
        }
    \end{flushleft}

    \blankfootnote{Source: \href{https://www.futura-sciences.com/planete/actualites/developpement-durable-emissions-co2-liees-internet-polluent-autant-avion-43802/}{Futura Sciences}}
\end{frame}

\begin{frame}
    \begin{flushleft}
        \textit{
            Je fais du vélo le matin, je voudrais ne pas respirer la poussière générée
            par ces datacenters fous fonctionnant si mal avec des algorithmes,
            car on se donne entre les mains un pouvoir qu'il ne maîtrise pas.
        }
        \begin{flushright}
            \tiny{---Julien Pauli}
        \end{flushright}
    \end{flushleft}

    \blankfootnote{Source: \href{http://blog.jpauli.tech/a-new-move-in-my-career/}{Julien Pauli blog}}
\end{frame}

\begin{frame}
    \begin{flushleft}
        \textit{
            Si vous voulez être un meilleur programmeur, apprenez un autre language que
            le PHP. Si possible un language fonctionnel.
        }
        \begin{flushright}
            \tiny{---Larry Garfield}
        \end{flushright}
    \end{flushleft}

    \blankfootnote{Source: \href{https://www.youtube.com/watch?v=3mcGzBnSm1c}{Nashville PHP Meetup: 2020-05-12. Never* Use Array with Larry Garfield}}
\end{frame}

\begin{frame}{loophp/collection}{A propos}
    \begin{itemize}[<+->]
        \item Commencé en Août 2019
        \item Entièrement paresseuse
        \item Pas de dépendances
        \item $\pm$ 85 méthodes/opérations disponibles
        \item Testée et typée
    \end{itemize}
\end{frame}

